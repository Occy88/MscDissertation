%ABSTRACT

%what am I writing about and what is it's usefullnes/point of it.
Automated planning has been a topic of research since the 1960s.
It came into fruition due to the desirability of accomplishing tasks autonomously, which is a problem in almost all
activity-based domains. Tasks are solved by an ordered set of actions over a domain. The job of formally defining
actions, their conditions and effects, becomes increasingly difficult and prone to error as domains evolve to be ever
more complex. For such reasons, in recent years, researchers have leveraged progress in Machine and Deep Learning to
automate the generation of action models from synthetic and real-world data. Machine and Deep Learning techniques
are evolving at an increasingly fast rate, but their integration with action model generation models has been difficult
as no clear translation layer exists between domain languages and common learning libraries.
Compounded with the multitude of existing domain languages new issues have manifested such
as a lack of testing and use-able implementations of current reconstruction approaches.

% Summary of what was done in this Disertation
This dissertations provides an overview of the current state of the art in reconstruction of action models, discusses
core issues that have hampered rapid development and implementation of more powerful reconstruction techniques available
today, and finally proposes a framework standardising future development of action-model reconstruction models.
A proof of concept project is then tested with an existing learning model using Markov Logic Networks,
finally we evaluate our results for metrics such as development time, development ease, modularity, and more.

% Summary of results/observations.
We have found that the proposed framework can work in the simulated scenario and provides a good foundation for more
complex learning model implementations. As it is based in Python is highly modular and provides relatively simple integration with
all modern learning techniques. This framework also lowers the entry barrier for new researchers in the field as there
is no longer a need to question language format or how to implement it in python as most existing
languages can be supported in the future. We also provide an example project for new users to experiment with as a
means to understand the structure and workings of the framework, as well as have an initial taste of modern
Domain based languages whilst remaining in the comfort of using Python.




% quick intro


% what I'm doing

% what i've done\newline
% software eng


% challenges I've faced

% challenges/next term
