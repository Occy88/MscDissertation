\chapter{Introduction}

\cite{ReviewActionModels:article}

% what is the problem space
% Some examples
% How do the above examples tie in with issues faced
% What needs to1 be answered in order to create the system that we propose (rethorical)
% Which of these has been attempted how did it work out, -> hence us doing it.

% What is automated planning/ reconstruction
This dissertation focuses on Automated Planning (AP), a well-established branch of artificial intelligence that
concerns itself with the generation of strategies or plans typically fulfilled by intelligent agents such as
(Warehouse robots, Assembly line robots, UAV's, MAV's, home roomba type robots).

A plan or plan trace is defined by a series of actions meant for the execution on an initial state space
transitioning it to a pre-defined goal state. Executing such a plan accomplishes a task such as the evasion
of an obstacle or acquisition of an object.
%languages
The laws governing a state space (action-models, objects, relations) are usually interpreted by a formal language.
Many such languages exist today (PDDL, PPDDL, STRIPS, OCL, ADL), each created to improve expressiveness, ease of use,
or add desired functionality.
Such languages were introduced as an attempt to standardize the process of defining a domain and solving planning problems.
Languages are similar in their structure, a domain and a problem must be defined, which is then used by a compatible
solver to find a solution for the problem in said domain.

Knowledge Engineering (KE) for AP is the process that deals with the overall generation of domain models from existing
data (plans, actions, states), including the acquisition, formulation, validation and maintenance of knowledge for the
model \cite{AutomatedKETools:inproceedings}.

A common situation in KE is that the encoding of the domain left to a domain expert. Often being a laborious and complex task,
it makes future maintenance difficult whilst also being prone to human induced error, testing is also a non-trivial task.
This issue compounded with the ever increasing complexity of real-world scenarios thanks to advancements in computation
and commercial need for automation, means that the challenge of engineering domain models is ever-more challenging.
In an effort to tackle this issue and assisting domain experts in modeling domains, new systems are constantly
being researched and developed for the automatic acquisition of AP models.

Existing state of the art planning algorithms such as AMAN, LOPE, ARTUE, ARMS and more \cite{ReviewActionModels:article}
rely on a multitude of different learning based approaches, such as reinforcement learning, MAX-SAT,
supervised learning and transfer learning. Each of these algorithms also rely on specific input
data and output action models in different languages. This added complexity makes it difficult for anybody without
expertise in a particular implementation, of using newly developed techniques. It has also been observed that working
examples of implementations are difficult to obtain and test due to the lack of standard data generation techniques
often provided by online resources which no longer exist.

The variance in development standards and models produced in this field, demonstrates a clear lack of organization and
management of existing algorithms, techniques and development standards.
An argument could be made that a lack of commercial appropriation of such niche techniques
(in industry) made the development of libraries comparable to Tensorflow or Pytorch overly laborious and
time-wise uneconomical for individuals. However the lack of such robust and readily available code makes teaching and
implementation or even experimentation of existing techniques unnecessarily complex often causing 3rd parties to avoid
implementations by researchers. We also have to consider the constant evolution of programming languages.
As time goes on and problems become more complex, hence there is a constant shift towards more abstract/higher level languages.
This is thanks to their inherently improved legibility and development speed often at the cost of execution speed.
For such reasons most complex algorithms and data structures are usually written in lower level languages
such as C/C++ and then wrapped by most popular higher level languages for easy implementation such as GO and Python,
this has not been the case with Domain languages as language standards created tens of years ago are still in common use today.


As a result of issues discussed, we can observe that teaching bodies or inexperienced users often resort implementing
planning problems from scratch, using custom data structures and algorithmic implementations in higher level languages
such as python due to their comfort; rather than implementing a domain using a Domain language and a solver in
their preferred higher level language. This situation is fine for the individual case, hwever it also translates
to more advanced use cases as people grow used to the tools they are taught.

% What are the key objectives from the above (to solve/analyse/future improvement)
% What was the goal and where does it stand int he dissertation
% How are results evaluated (where in the disseratation/what was done) / any difficulties
% 
%Objectivesz
% Methods
% Results
% Conclusions
We ask ourselves, what needs to be done in order to tackle these issues.
The two most important tasks that we believe need to be tackled are as follows.
First is that a solution needs to be useful for researchers in the field,
mainly saving them time and effort when developing new ideas.
This means the solution needs to be highly adaptable/modular to any environment allowing for constant improvement and extension.
It also has to aid in mundane tasks such as data generation and management for testing and implementation.
This allows for more accessible and robust and verifiable implementations.
Examples exist for machine learning, such as importing of pre-processed datasets (mnist etc...) directly from a package.
Secondly the system needs to be simple and robust for users alien to the field.
This requires good documentation, and high level access to existing algorithms without the need to open the guts of a particular implementation.


Given the above we propose a framework with an initial proof of concept built in python,
integrating a custom implementation of a single algorithm for reconstruction of action-models using Markov Logic Networks,
and test this implementation against existing data.
Our objectives for the framework are as follows:
% Todo: review
\begin{enumerate}
    \item Support for parsing various domain languages to a python standard.
    \item Synthetic generation of training data.
    \item Simplified implementation of learning models.
    \item Extendable modules for conversion of training output to a standard action model.
    \item Testing framework for learned action models.
\end{enumerate}
