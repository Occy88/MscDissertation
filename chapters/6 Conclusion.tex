\chapter{Discussion}
In this thesis the main objective was to create a python based framework in the Planning domain to enable us to create and train a model for reconstruction action models from plan traces using existing techniques such as MLN's.
The reason for creating this software was to address the lack of tools available to experiment and develop with Planning languages in our case PDDL.
We chose our aims based on what functionality would enable new and experienced users to use Planning languages programming languages they are already acquainted with.
We found that Python being the most popular language with support for ML and DL libraries, had no serious support for Planning.
Based on these findings we developed a framework with modularity in mind to provide a solid baseline for people working in this field to build on.
As we publish these tools on opensource libraries we expect new adopters to be able to provide their own additions to the platform.

We found that thanks to the tools we provided picking up new domains or designing our own domains and working with them is much easier than before.
The adoption barrier to the field has been significantly reduced.
Thanks to the introduction of our tools, available solvers or parsers would have to be found online, as discussed these often provide unclear documentation and instructions, often breaking when initially being used if the environment is not set-up correctly.
Now thanks to the standards in place, it is as simple as importing a package and calling a function to generate solutions on a domain and problem.
We can observe from the sample code provided that a normal workflow in the planning field using out tools now looks similar to any modern ML framework used by Data Scientists.
This is a stark contrast to the standard of creating custom data structures with any research project in the field simply for interpreting or simulating domains.

In the future, as with every framework, support needs to be included for more solvers and languages.
This is the main limitation of the current state of the software we produced, we simply did not have the resources to expand support beyond the minimum specification as the goal was to provide the foundation upon which people can build.


%limitations
% arguments & key takeaways
% answered main question
% relevant recommendation
% contributions of the research done
